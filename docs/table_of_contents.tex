% Зачем: Содержание пишется нормальным шрифтом, по центру всеми заглавными буквами
% Почему: Пункт 2.2.7 Требований по оформлению пояснительной записки.
%\renewcommand \contentsname {\centerline{{\normalsize{\normalfont{СОДЕРЖАНИЕ}}}}}

\renewcommand\contentsname{\hfill СОДЕРЖАНИЕ \hfill}
\renewcommand\cfttoctitlefont{\normalfont\normalsize}% use "\bfseries" if you want it in bold
\renewcommand{\cftaftertoctitle}{\hfil}
%\setlength\cftbeforetoctitleskip{0.2cm}

\def\formatsection{\MakeUppercase}

\makeatletter
\let\oldcontentsline\contentsline
\def\contentsline#1#2{%
  \expandafter\ifx\csname l@#1\endcsname\l@section
    \expandafter\@firstoftwo
  \else
    \expandafter\@secondoftwo
  \fi
  {%
	  \oldcontentsline{#1}{\MakeUppercase{#2}}%
  }{%
	\oldcontentsline{#1}{#2}%
  }%
}
\makeatother

%% Зачем: Не захламлять основной файл
%% Примечание: \small\selectfont злостный хак, чтобы уменьшить размер шрифта в ToC
{
\normalfont
\tableofcontents
\newpage
}
